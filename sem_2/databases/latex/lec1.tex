\documentclass[a4paper]{article}

\usepackage[utf8]{inputenc}
\usepackage[T1]{fontenc}
\usepackage{textcomp}
\usepackage[english]{babel}
\usepackage{amsmath, amssymb}
\title{Relational algebra}
\author{Andrii Stadnik}

\begin{document}
	\maketitle
	\tableofcontents
	\section{Lecture 1}
	\subsection{Intro}

	The hardest - to normalize. There are a lot of normal forms: 6. DB has to
	satisfy them. High forms are not used (5, 6). The main idea: we have a big
	table, and then we split them into the smallest ones. But the problem is
	that you have to connect them, which is hard. 

	There is relational algebra first, and then based on it relational DBs were
	created.

	The main thing in relational algebra is \emph{relations}.

	\subsection{Sets}

	What we will define first is \emph{nominative sets}. Those are like
	dictionaries in programming languages. 
	Definitions:
	\begin{itemize}
		\item It's a finite set of pairs ${(n_1, v_1), \ldots, (n_k, v_k)}$,
			where $n_i \in N, v_i \in V, n_i \neq n_j$, if $i \neq j$, $i, j =
			1..k $.
		\item It's a finite mapping from space N to the space V.
	\end{itemize}
	We will write it as $\{n_1: v_1, \ldots, n_k, v_k\}$.
	Example: $\{last_name:'Smith', birthday:3.9.1990\}$
	\subsection{Tuples}
	Let's define some things:
	\begin{description}
		\item[A] Set of attributes' names. $a_1$
		\item[Dom] Set of the atomic data
		\item[Schema S] Finite set of attributes' names $a_1, a_2, \ldots, $
	\end{description}


\end{document}
